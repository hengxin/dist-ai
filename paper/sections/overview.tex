% overview.tex

%%%%%%%%%%%%%%%%%%%%%%%%%%%%%%
\section{Overview}  \label{section:overview}


\subsection{流程}\label{ss:invgen}

\begin{itemize}
  \item 通过类型分析,得到\tlaplus{} 规约中的变量类型
  \subitem Apalache工具, Json格式
  \item 根据转换规则,将每个变量用一组relation表示
  \item 对\tlaplus{}的trace,将每个状态中变量的值替换为relation的值,得到用关系表示的trace
  \item 将用关系表示的trace发送给DistAI,得到候选的不变式
  \subitem DistAI只能生成没有$\exists$的公式
  \subitem 由于\tlaplus{}的变量的数据结构的复杂性,每个变量可能需要一组关系才能表示,这样的关系可能整体考虑才有意义,但是DistAI
    的枚举没有这样的保证
  \subitem 用关系表示时会引入局部变量
  \item 将由relation组成的不变式中的relation替换成\tlaplus{}中变量的符号,得到用\tlaplus{}变量组成的候选不变式
  \item 将候选的不便式通过Apalache通过检验
\end{itemize}

\subsection{转换规则}\label{ss:transrule}


\tlaplus{} 的类型系统定义: 

$
  t \doteq Cons 
  \mid Bool 
  \mid Int 
  \mid t \rightarrow t 
  \mid Set(t) 
  \mid t\times\cdots\times t 
  \mid \lbrack nm_1:t,\cdots,nm_k:t\rbrack
$

将每一个变量$v$,用一组关系$RS_v$表示:

\begin{itemize}
  \item 若$v.type = Cons\mid Int\mid Bool$, 定义关系$R_v(\_):v=x \leftrightarrow R_v(x)$. 
        则$RS_v =R_v$.
  \item 若$v.type = Set(t)$,定义$R_v(\_):x\in v \leftrightarrow R_v(x)$
    \subitem 若$t\in tb$, 则$RS_v = R_v$
    \subitem 若$t \notin tb$, 则$RS_v = R_v(x)\wedge RS_x$($x$是$t$类型的一个占位符)
  \item 若$v.type = T_1\times T_2\times \cdots \times T_k$, 则定义$k$个关系$R_i(\_): v[i]=x \leftrightarrow  R_i(x)$。
        对每个关系$R_i$, 定义$RS_i$:
    \subitem 若$T_i\in tb$, 则$RS_i=R_i$
    \subitem 若$T_i \notin tb$, 则$RS_i=R_i(x) \wedge RS_x$($x$是$T_i$类型的占位符)
      \\
        则$RS_v = RS_1\wedge RS_2\wedge \cdots \wedge RS_k$
  \item 若$v.type = T_1\rightarrow T_2 \rightarrow \cdots \rightarrow T_k$, 则定义$(k+1)$元关系
        $R_v(\_,\cdots\_):v(x_1,\cdots,x_k) = x_{k+1} \equiv R_v(x_1,\cdots,x_k.x_{k+1})$。其中,
    \subitem 若$T_i\in \{Cons,Int,Bool\}$, 则$x_i$为具体的值, 且$RS_i = true$
    \subitem 若$T_i \notin tb$, 则$x_i$为$T_i$类型的占位符, 且$RS_i = RS_{x_i}$
    \\
    则$RS_v = R_v(x_1,\cdots,x_k,x_{k+1}) \wedge RS_{x_1} \wedge RS_{x_2} \wedge \cdots \wedge RS_{x_{k+1}}$
  \item 若$v.type = [nm_1:T_1,\cdots,nm_k:T_k]$, 由于record类型的各个项之间没有顺序关系, 所以可以按照某种规则(如字典序)
      将$\{nm_1,\cdots,nm_k\}$ 进行排序. 假设已经进行了排序, 那么record类型的规则和$k$元tuple相同。定义$k$个关系$R_i(\_)$: 
      $v.nm_k = x \leftrightarrow R_k(x)$. 对于每个关系$R_i$, 定义$RS_i:$
    \subitem 若$T_i \in \{Int,Bool,Cons\}$, 则$RS_i = R_i$
    \subitem 若$T_i \notin \{Int,Bool,Cons\}, 则RS_i = R_i(x)\wedge RS_x$
    \\
  则$RS_v = RS_1\wedge RS_2\wedge \cdots \wedge RS_k$

\end{itemize}


%%%%%%%%%%%%%%%%%%%%
\subsection{Sampling \tlaplus{} Traces} \label{ss:sampling}

%%%%%%%%%%%%%%%%%%%%
\subsection{Enumerating Invariants} \label{ss:enumerating}

\begin{itemize}
  \item directed by syntax of \tlaplus
  \item restricting terms, operations, $\ldots$
\end{itemize}
%%%%%%%%%%%%%%%%%%%%
\subsection{Validating Inductive Invariants} \label{ss:validating}

\begin{itemize}
  \item using Apalache (modified for validating fols with quantifiers)
  \item using \cite{ProofAutomation:PhDThesis2014}
\end{itemize}
%%%%%%%%%%%%%%%%%%%%%%%%%%%%%%